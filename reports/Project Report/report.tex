\documentclass[11pt, a4paper, twoside, openright]{report}
\usepackage{float} % lets you have non-floating floats
\usepackage{url} % for typesetting urls
%  We don't want figures to float so we define
\newfloat{fig}{thp}{lof}[chapter]
\floatname{fig}{Figure}


\title{Genetic Programming for Antarctic Ice Sheet Modelling}
\author{Samuel Mata}
\usepackage[font, ecs]{vuwproject} 
\supervisors{Dr. Bach Nguyen, Dr. Bing Xue}
\otherdegree{Bachelor of Science with Honors in Artificial Intelligence}
\date{}

\begin{document}

% Make the page numbering roman, until after the contents, etc.
\frontmatter

%%%%%%%%%%%%%%%%%%%%%%%%%%%%%%%%%%%%%%%%%%%%%%%%%%%%%%%

\begin{abstract}
  This project aims to investigate and evaluate the use of Genetic
  Programming \textit{(GP)} and Evolutionary Learning techniques for
  the long-term modelling of Antarctic Ice Sheet measurements. 
\end{abstract}

%%%%%%%%%%%%%%%%%%%%%%%%%%%%%%%%%%%%%%%%%%%%%%%%%%%%%%%

\maketitle

%\tableofcontents

% we want a list of the figures we defined
%\listof{fig}{Figures}

%%%%%%%%%%%%%%%%%%%%%%%%%%%%%%%%%%%%%%%%%%%%%%%%%%%%%%%

\mainmatter

%%%%%%%%%%%%%%%%%%%%%%%%%%%%%%%%%%%%%%%%%%%%%%%%%%%%%%%

\section*{1. Problem Statement}

The changing conditions of Antarctica's Ice Sheets are a significant
factor in the changing global climate, particularly with respect to
rising sea levels [REFERENCE]. In this context, the ability to
accurately model the long-term behaviour of these ice sheets is
of great importance.

\section*{2. Motivations}

Current modelling of Antarctic Ice Sheet measurements - such as
that being undertaked at Victoria University's Antarctic Research
Center \textit{(ARC)} - is commonly based on the use of traditional
statistical models. These models are complex and time-consuming to 
use, limiting the practical scope of the predictive capabilities. 
The use of machine learning models shows promise as an approach due to the
improved computational efficiency of these predictive systems. 
Specifically, the use of Genetic Programming \textit{(GP)} and
Evolutionary Learning techniques provide the greatest potential
due to the increased explainability these methods provide.

\section*{3. Goals}

The primary goal of this project is to investigate the
application of Genetic Programming and Evolutionary Learning
techniques on the problem of Antarctic Ice Sheet modelling.
This will involve several stages:
\begin{enumerate}
\item \textbf{An initial Exploratory Data Analysis \textit{(EDA)}} of
  the data provided by the ARC, to greater understand the
  the nature of the problem and the data. \textit{( 2 weeks)}
  
  This will involve the use of univariate, bivariate, 
  and multi-variate analysis to identify the distribution,
  shape, and relationships of the features.

  Variables should also be analysed in terms of their 
  spatial and temporal distributions.
  
  Finally, potential feature engineering opportunities
  should be explored to identify potential improvements.
\item \textbf{Development of target-independant models}
  to predict target variables using advanced machine learning principles.
  At this stage each target should be predicted independently, without
  usage of other predicted values.
  \textit{(~6 weeks)}

  Each model should explore implementing a different
  approach, before analysing and evaluating the results
  against previous iterations.
\item \textbf{Development of target-dependant models} 
  for similar predictions, but taking into account the possible 
  relationships between target variables.
  \textit{(~6 weeks)}

  Each model should explore implementing a different
  approach, before analysing and evaluating the results
  against previous iterations.
\item \textbf{Evaluation of model interpretability and explainability},
  targetting the explainability and interpretability of
  the predicted outputs. \textit{(~4 weeks)}

  This will involve analysing and visualising the
  underlying mathematical expressions evolved by the model,
  and checking these against current scientific
  understanding of the problem.
\item \textbf{Further evaluation and improvement of the most effective model}, to
  determine the potential for improvement over traditional
  methods. \textit{(~4 weeks)}

  Evaluation should consider many measures, particularly the
  computational efficiency of the model, the accuracy of the
  predictions, and the interpretability of the results.
\end{enumerate}
Project milestones should be set for each of these stages
to ensure that the project remains on track. It is important
to note the potential for these goals to change as the 
direction of the project becomes clearer.

\section*{4. Evaluation}

\section*{5. Resource Requirements}

No external or additional resources are required for the project as all tooling is publicly and freely accessible.

%%%%%%%%%%%%%%%%%%%%%%%%%%%%%%%%%%%%%%%%%%%%%%%%%%%%%%%
\backmatter
%%%%%%%%%%%%%%%%%%%%%%%%%%%%%%%%%%%%%%%%%%%%%%%%%%%%%%%

%\bibliographystyle{ieeetr}
\bibliographystyle{acm}
\bibliography{sample}
\end{document}
