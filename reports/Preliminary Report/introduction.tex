\chapter{Introduction}\label{C:introduction}

\section{Project Outline}

This project, as outlined in the proposal \textit{(see \Cref{C:project_proposal})}, aims to investigate and evaluate the use of genetic programming \textit{(GP)} techniques in the context of Antarctic ice sheet modelling. The successful implementation of such techniques would provide a faster and less computationally demanding alternative to current physics-based simulations. The use of genetic programming  techniques would also lend interpretability to the predicted outputs of the model, allowing for comparison and evaluation with current simulation methods.

\section{Deviations from Original Proposal}

Current progress has been focused primarily on the exploratory data analysis \textit{(EDA)} of the  dataset provided by Victoria University’s Antarctic Research Center (ARC) . This analysis has taken longer than initially expected, largely due to some unforeseen intricacies in the dataset \textit{(see \Cref{S:UVA})} as well as my unfamiliarity with antarctic ice sheet modelling. Such challenges were expected, with the timeline allotted for each stage of development allowing for some variability. Therefore, this delay does not alter the research problem being addressed and the plan for development going forwards remains largely unchanged.